\chapter{Аналитическая часть}
В данном разделе будут описаны задача коммивояжёра, а также алгоритм полного перебора и муравьиный алгоритм для решения данной задачи.


\section{Постановка задачи}
\textit{Задача коммивояжёра} -- важная задача транспортной логистики \cite{com},
отрасли, занимающейся планированием транспортных перевозок.
Коммивояжёру, чтобы распродать нужные и не очень нужные в хозяйстве товары,
следует объехать n пунктов и в конце концов вернуться в исходный пункт.
Требуется определить наиболее выгодный маршрут объезда.
В качестве меры выгодности маршрута (точнее говоря, невыгодности)
может служить суммарное время в пути, суммарная стоимость дороги, или,
в простейшем случае, длина маршрута.



\section{Описание алгоритмов}

\subsection{Алгоритм полного перебора}

Метод полного перебора, по-другому именуемый методом грубой силы, является простым, логичным и широко используемым
математическим методом. Он применим во многих, если не во всех,
областях математики: задача коммивояжера также не является исключением.
Идея brute force предельно проста: перебираются всевозможные
решения и из них выбирается решение (или множество решений) отвечающее
условию задачи.

В задаче коммивояжера, соответственно, требуется из всевозможных
вариантов объезда пунктов выбрать маршрут, занимающий кратчайшее время
(или минимальный по стоимости маршрут).

Огромным преимуществом метода полного перебора перед другими
методами решения задачи коммивояжера является гарантированность
нахождения наилучшего маршрута. Другие методы советуют лишь
«хороший» маршрут, который совсем не обязательно является лучшим. Кроме
того, к достоинствам метода относится простота его программной реализации.

Однако, в связи с наличием огромного недостатка, метод полного
перебора крайне редко используется на практике. Этим недостатком является
временная сложность алгоритма. Асимметричная задача коммивояжера с n
посещаемых пунктов требует при полном переборе рассмотрения (n-1)! туров,
а факториал  растет невероятно быстро. Поэтому метод полного перебора может применяться только для задач
малой размерности (при рассмотрении до двух десятков посещаемых
пунктов).


\subsection{Муравьиный алгоритм}

В то время как простой метод перебора всех вариантов чрезвычайно
неэффективный при большом количестве городов,
эффективными признаются решения, гарантирующие получение
ответа за время, ограниченное полиномом от размерности задачи.

В основе алгоритма лежит поведение муравьиной колонии \cite{ulianov} -- маркировка более удачных
путей большим количеством феромона.
Рассмотрим биологическую модель поведения такой колонии.

В реальном мире муравьи (первоначально) ходят в случайном порядке и по нахождении
продовольствия возвращаются в свою колонию, прокладывая феромонами тропы.
Если другие муравьи находят такие тропы, они, вероятнее всего, пойдут по ним.
Вместо того, чтобы отслеживать цепочку, они укрепляют её при возвращении,
если в конечном итоге находят источник питания. Со временем феромонная тропа
начинает испаряться, тем самым уменьшая свою привлекательную силу. Чем больше
времени требуется для прохождения пути до цели и обратно, тем сильнее испарится
феромонная тропа. На коротком пути, для сравнения, прохождение будет более быстрым,
и, как следствие, плотность феромонов остаётся высокой.
Если бы феромоны не испарялись, то путь, выбранный первым,
был бы самым привлекательным. В этом случае, исследования пространственных
решений были бы ограниченными. Таким образом, когда один муравей находит
(например, короткий) путь от колонии до источника пищи, другие муравьи,
скорее всего пойдут по этому пути, и положительные отзывы в конечном итоге
приводят всех муравьёв к одному, кратчайшему, пути. Этапы работы муравьиной
колонии представлены на рис. 1.1.

\img{100mm}{ant}{Работа муравьиной колонии}

Процесс поиска кратчайшего пути от колонии до источника питания (рис. 1.1):

\begin{enumerate}
	\item первый муравей находит источник пищи (F) любым способом (а), а затем возвращается к гнезду (N), оставив за собой тропу из феромонов (b);
	\item затем муравьи выбирают один из четырёх возможных путей, затем укрепляют его и делают привлекательным;
	\item муравьи выбирают кратчайший маршрут, так как феромоны с более длинных путей быстрее испаряются.
\end{enumerate}


Вероятность перехода из вершины i в вершину j определяется по формуле \ref{form:way}.

\begin{equation}\label{form:way}
	p_{i,j}={\frac {(\tau_{i,j}^{\alpha })(\eta_{i,j}^{\beta })}{\sum (\tau_{i,j}^{\alpha })(\eta_{i,j}^{\beta })}}
\end{equation}

где $ \tau_{i,j} - $ расстояние от города i до j;

$\eta_{i,j} - $количество феромонов на ребре ij;

$\alpha - $ параметр влияния длины пути;

$\beta - $ параметр влияния феромона.

Уровень феромона обновляется в соответствии с формулой \ref{form:eva}


\begin{equation}\label{form:eva}
	\tau_{i,j}=(1-\rho )\tau_{i,j}+\Delta \tau_{i,j},
\end{equation}

где $\rho$ - \text{доля феромона, которая испарится;}

$\tau_{i,j}$ - \text{количество феромона на дуге ij;}

$\Delta \tau_{i,j}$ - количество отложенного феромона, вычисляется по формуле \ref{form:add1}.

\begin{equation}\label{form:add1}
	\Delta \tau_{i,j}= \tau_{i,j}^0 + \tau_{i,j}^1 + ... + \tau_{i,j}^k
\end{equation}

где k - количество муравьев в вершине графа с индексами i и j.


Описание поведения муравьев при выборе пути.

\begin{itemize}
	\item Муравьи имеют собственную «память».
	      Поскольку каждый город может быть посещён только один раз, то у каждого муравья есть список уже посещенных городов - список запретов.
	      Обозначим через $J_{ik}$ список городов, которые необходимо посетить муравью $k$, находящемуся в городе $i$.
	\item Муравьи обладают «зрением» - видимость есть эвристическое желание посетить город $j$ , если муравей находится в городе $i$.
	      Будем считать, что видимость обратно пропорциональна расстоянию между городами.
	\item Муравьи обладают «обонянием» - они могут улавливать след феромона, подтверждающий желание посетить город $j$ из города $i$ на основании опыта других муравьёв.
	      Количество феромона на ребре $(i,j)$ в момент времени $t$ обозначим через  $\tau_{i,j} (t)$.
	      % \item На этом основании мы можем сформулировать вероятностно - пропорциональное правило, определяющее вероятность перехода $k$-ого муравья из города $i$  в город $j$.
	\item Пройдя ребро $(i,j)$ , муравей откладывает на нём некоторое количество феромона, которое должно быть связано с оптимальностью сделанного выбора.
	      Пусть $T _{k} (t)$ есть маршрут, пройденный муравьем $k$ к моменту времени $t$ , $L _{k} (t)$ - длина этого маршрута, а $Q$ - параметр,
	      имеющий значение порядка длины оптимального пути. Тогда откладываемое количество феромона может быть задано формулой \ref{form:add}.

\end{itemize}



\begin{equation}\label{form:add}
	{\displaystyle \Delta \tau_{i,j}^k={\begin{cases}Q/L_{k}, & {\mbox{если k-ый мурваей прошел по ребру ij;}}\\0,&{\mbox{иначе}}\end{cases}}}
\end{equation}

где Q - количество феромона, переносимого муравьем.
\section*{Вывод} 
В данной работе стоит задача реализации алгоритмов для решения задачи коммивояжёра. Входными данными будет являтся матрица $M$ расстояний между городами размера  $n * n$, где n - количество строк и столбцов, а $M_{i,j}$ - расстояние из города с индексом i в город с индексом j. Выходными данными будет являтся целое число - кратчайший маршрут, а также массив содержащий номера городов кратчайшего маршрута. В связи с ограничениями накладываемыми на ПО, входное число $n$ должно быть корректным, а также все расстояния должны быть целочисленными. Необходимо дать теоретическую оценку алгоритма полного перебора и муравьиного алгоритма.
