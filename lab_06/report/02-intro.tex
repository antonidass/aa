\chapter*{Введение}
\addcontentsline{toc}{chapter}{Введение}

Одна из самых известных и важных задач транспортной логистики (и комбинаторной оптимизации) – задача коммивояжёра или "задача о странствующем торговце". Суть задачи сводится к поиску оптимального (кратчайшего, быстрейшего или самого дешевого) пути, проходящего через промежуточный пункты по одному разу и возвращающегося в исходную точку. К примеру, нахождение наиболее выгодного маршрута, позволяющего коммивояжёру посетить со своим товаром определенные города по одному разу и вернуться обратно. Мерой выгодности маршрута может быть минимальное время поездки, минимальные расходы на дорогу или минимальная длина пути. В наше время, когда стоимость доставки часто бывает сопоставима со стоимостью самого товара, а скорость доставки - один из главных приоритетов, задача нахождения оптимального маршрута приобретает огромное значение.   

Муравьиный алгоритм - один из эффективных полиномиальных алгоритмов для нахождения приближенных решений задачи коммивояжёра, а также решения аналогичных задач поиска маршрутов на графах. Суть подхода заключается в анализе и использовании модели поведения муравьев, ищущих пути от колонии к источнику питания, и представляет собой метаэвристическую оптимизацию.

Целью данной работы является изучение следующих алгоритмов решения задачи коммивояжёра: 
\begin{itemize}
	\item муравьиный алгоритм;
	\item наивный алгоритм;
\end{itemize} 

В рамках выполнения работы необходимо решить следующие задачи:
\begin{itemize}
	\item рассмотреть и изучить подходы к решению задачи коммивояжёра;
	\item привести схемы реализации алгоритмов решения задачи коммивояжёра;
	\item описать структуру разрабатываемого ПО;
	\item определить средства программной реализации;
	\item протестировать разработанное ПО;
	\item привести сведения о модулях программы;
	\item определить требования к ПО;
	\item провести экспериментальные замеры временных характеристик реализованных алгоритмов;
	\item на основании проделанной работы сделать выводы и подготовить отчет.
\end{itemize}