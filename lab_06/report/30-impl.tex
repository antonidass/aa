\chapter{Технологическая часть}

В данном разделе приведены средства реализации и листинг алгоритмов.

\section{Средства реализации}

Для разработки ПО был выбран язык Java, поскольку он предоставляет разработчику широкий спектр возможностей и позволяет разрабатывать кроссплатформенные приложени, а также предоставляет большой набор инструментов для работы с многопоточностью. В качестве среды разработки использовалась Visual Studio Code. \cite{code}

\section{Листинг кода}


\begin{lstinputlisting}[
	caption={Реализация алгоритма полного перебора},
	label={lst:first},
	style={javalang},
	linerange={6-48}
]{../src/Brute.java}
\end{lstinputlisting}



\begin{lstinputlisting}[
	caption={Реализация муравьиного алгоритма},
	label={lst:second},
	style={javalang},
	linerange={9-173}
]{../src/Ant.java}
\end{lstinputlisting}


\section{Тестирование функций}

В таблице~\ref{tabular:test_rec} приведены тесты для функций, реализующих алгоритм кодирования строки. Тесты пройдены успешно.

\begin{table}[h!]
	\begin{center}
		\begin{tabular}{c@{\hspace{7mm}}c@{\hspace{7mm}}c@{\hspace{7mm}}c@{\hspace{7mm}}c@{\hspace{7mm}}c@{\hspace{7mm}}}
			\hline
			Количество городов & Матрица расстояний & Ожидаемый результат \\ \hline
			\vspace{4mm}
			4
			&
			$\begin{pmatrix}
				0 & 10 & 15 & 20 \\ 
				10 & 0 & 35 & 25 \\
				15 & 35 & 0 & 30 \\
				20 & 25 & 30 & 0 
			\end{pmatrix}$ 
			&
				80
				\\
			\vspace{2mm}
			\vspace{2mm}
			4 &
			$\begin{pmatrix}
				0 & 8 & 9 & 7 \\ 
				8 & 0 & 4 & 4 \\
				9 & 4 & 0 & 2 \\
				7 & 4 & 2 & 0
			\end{pmatrix}$ 
			&
			21
		\end{tabular}
	\end{center}
	\caption{\label{tabular:test_rec} Тестирование функций}
\end{table}

 
\section*{Вывод}

Правильный выбор инструментов разработки позволил эффективно реализовать алгоритмы и выполнить исследовательский раздел лабораторной работы.
