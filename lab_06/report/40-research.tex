\chapter{Исследовательская часть}

В данном разделе будет произведено сравнение алгоритма полного перебора для решения задачи коммивояжёра и муравьиного алгоритма, а также будет приведена демонстрация работы программы.



\section{Технические характеристики}

\begin{itemize}
	\item Операционная система: Windows 10. \cite{windows}
	\item Память: 16 GiB.
	\item Процессор: Intel(R) Core(TM) i7-4700HQ CPU @ 2.40GHz. \cite{intel}
\end{itemize}


\section{Временные характеристики}

Для сравнения возьмем 8 матриц расстояний размерностью: [3, 4, 5, ... 10]. Возспользуемся усреднением массового эксперимента. Для этого сложим результат работы алгоритма N раз (N >= 10), после чего поделим на N. Тем самым получим достаточно точные характеристики времени. Результат сравнения алгоритма полного перебора и муравьиного алгоритма представлен на рис 4.1.

\img{120mm}{disc}{Результат сравнения}

\clearpage

\section{Параметризация мурвьиного алгоритма}
В муравьином алгоритме вычисления производятся на основе настраиваемых параметров. Рассмотрим матрицу смежностей размерностью $10\times10$.


\begin{table}[ht]
	\centering
	\caption{Матрица смежностей}
	\label{table:matrix}
	\begin{tabular}{ | l | l | l | l | l | l | l | l | l | l | l |}
		\hline
		0 & 0    & 1    & 2    & 3    & 4    & 5    & 6    & 7    & 8    & 9    \\ \hline
		0 & 0    & 1790 & 200  & 1900 & 63   & 1659 & 1820 & 1395 & 2382 & 649  \\ \hline
		1 & 1790 & 0    & 1573 & 2435 & 1515 & 714  & 892  & 2193 & 1590 & 1003 \\ \hline
		2 & 200  & 1573 & 0    & 833  & 392  & 2404 & 962  & 902  & 141  & 1123 \\ \hline
		3 & 1900 & 2435 & 833  & 0    & 2283 & 1652 & 2362 & 2262 & 1512 & 2166 \\ \hline
		4 & 63   & 1515 & 392  & 2283 & 0    & 1322 & 290  & 1305 & 2100 & 969  \\ \hline
		5 & 1659 & 714  & 2404 & 1652 & 1322 & 0    & 256  & 78   & 2236 & 2041 \\ \hline
		6 & 1820 & 892  & 962  & 2362 & 290  & 256  & 0    & 1180 & 1547 & 1279 \\ \hline
		7 & 1395 & 2193 & 902  & 2262 & 1305 & 78   & 1180 & 0    & 1640 & 1161 \\ \hline
		8 & 2382 & 1590 & 141  & 1512 & 2100 & 2236 & 1547 & 1640 & 0    & 2212 \\ \hline
		9 & 649  & 1003 & 1123 & 2166 & 969  & 2041 & 1279 & 1161 & 2212 & 0    \\ \hline
	\end{tabular}
\end{table}

\clearpage

\img{190mm}{k1}{Таблица коэффициентов. Часть 1}
\clearpage

\img{190mm}{k2}{Таблица коэффициентов. Часть 2}
\clearpage

\img{190mm}{k3}{Таблица коэффициентов. Часть 3}
\clearpage

\img{100mm}{k4}{Таблица коэффициентов. Часть 4}




\section*{Вывод}

В данном разделе было произведено сравнение количества затраченного времени вышеизложенных алгоритмов. В результате сравнения алгоритма полного перебора и муравьиного алгоритма по времени было получено, что при относительно небольших размерах матрицы смежности (от 2 до 8) алгоритм полного перебора работает быстрее (при размере 2 в 1000 раз). Однако при размере матрицы равном 9, время работы алгоритма полного перебора становится сопоставимым с временем работы муравьиного алгоритма. Более того при размерах матрицы больших 9, время работы алгоритма полного перебора начинает резко возрастать, и становится более чем в 10 раз медленнее  муравьиного алгоритма.

Исходя из проведенных исследований, можно сделать вывод, что муравьиный алгоритм решения задачи коммивояжёра выигрывает у алгоритма полного перебора при размерностях матриц равных 9 и более. В случае, когда количество вершин в графе меньше 9, лучше использовать алгоритм полного перебора.

