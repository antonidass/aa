\chapter{Конструкторская часть}
В данном разделе будут рассмотрены схемы вышеизложенных алгоритмов, описаны способы тестирования и определены структуры данных. 
\section{Разработка алгоритмов}

\img{170mm}{bruteforce}{Схема наивного алгоритма}


\clearpage
\img{190mm}{ant_upd}{Схема муравьиного алгоритма}


\section{Описание структур данных}
Исходя из условия задачи приходим к выводу о том, что для хранения информации о расстояниях между городами наиболее удобно использовать структуру данных - двумерный массив. Для хранения минимального пути необходимо использовать целочисленный одномерный массив.
  

\section{Структура ПО}
ПО будет состоять из следующих модулей:

\begin{itemize}
    \item Основной модуль;
    \item Модуль включающий в себя реализацию наивного алгоритма решения задачи коммивояжёра;
    \item Модуль включающий в себя реализацию муравьиного алгоритма;
    \item Модуль для работы с входными и выходными данными;
    \item Модуль вспомогательных функций;
\end{itemize}

\section{Описание способов тестирования}
В рамках данной лабораторной работы можно выделить следующие классы эквивалентности:
\begin{itemize}
    \item входными данными является ациклический ориентированный взвешенный граф;
    \item входными данными является циклический ориентированный взвешенный граф;
\end{itemize}

\section*{Вывод}

На основе теоретических данных, полученных из аналитического раздела, были построены схемы двух алгоритмов решения задачи коммивояжёра. Приведена структура ПО, описаны структуры данных.