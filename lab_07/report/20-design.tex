\chapter{Конструкторская часть}
В данном разделе будут рассмотрены схемы вышеизложенных алгоритмов, описаны способы тестирования и определены структуры данных. 
\section{Разработка алгоритмов}

\img{170mm}{brute}{Схема алгоритма поиска полным перебором}


\clearpage
\img{255mm}{binsearch}{Схема алгоритма с бинарным поиском}



\clearpage
\img{240mm}{freq}{Схема алгоритма с частотным анализом}
\clearpage



\section{Структура ПО}
На рисунке 2.4 представлена uml-диаграмма разрабатываемого ПО.
\img{120mm}{uml}{Структура программного обеспечения}


\section{Описание структур данных}
Описание используемых в программе типов данных:

\begin{itemize}
    \item \textbf{Dict} - класс содержащий в себе хеш-таблицу с ключами и значениями типа string, а также методы для работы над этой таблицей;
    \item \textbf{ArrayList<HashMap<String, Object>>} - тип данных, описывающий результат частотного анализа; 
    \item \textbf{HashMap<String, String>} - тип данных, описывающий ассоциативный массив с ключами типа $string$.
\end{itemize}
  


\section{Тестирование}
В рамках данной лабораторной работы можно выделить следующие классы эквивалентности:
\begin{itemize}
    \item входными данными является ключ и пустой словарь;
    \item входными данными является непустой словарь и ключ, который не существует в словаре;
    \item входными данными является непустой словарь и ключ, значение для которого в словаре определено;
\end{itemize}

\section*{Вывод}

На основе теоретических данных, полученных из аналитического раздела, были построены схемы трех алгоритмов поиска в ассоциативных массивах. Приведена структура ПО, описаны структуры данных и способы тестирования.