\chapter*{Введение}
\addcontentsline{toc}{chapter}{Введение}

Структура данных, позволяющая идентифицировать ее элементы не по числовому индексу, а по произвольному, называется словарем или ассоциативным массивом. Каждый элемент словаря состоит из двух объектов: ключа и значения. 
В жизни широко распространены словари, например, привычные бумажные словари (толковые, орфографические, лингвистические). В них ключом является слово-заголовок статьи, а значением — сама статья. Для того, чтобы получить доступ к статье, необходимо указать слово-ключ.

Особенностью ассоциативного массива является его динамичность: в него можно добавлять новые элементы с произвольными ключами и удалять уже существующие элементы. При этом размер используемой памяти пропорционален размеру ассоциативного массива. Доступ к элементам ассоциативного массива выполняется хоть и медленнее, чем к обычным массивам, но в целом довольно быстро.

С появлением словарей появилась нужда в том, чтобы уметь быстро
находить нужное значение по ключу. Со временем стали разрабатывать
алгоритмы поиска в словаре.

Целью данной работы является изучение следующих алгоритмов поиска в словаре: 
\begin{itemize}
	\item Поиск полным перебором;
	\item Бинарный поиск;
	\item Частотный анализ.
\end{itemize}

В рамках выполнения работы необходимо решить следующие задачи:
\begin{itemize}
	\item исследовать основные алгоритмы поиска в словаре;
	\item привести схемы реализации алгоритмов поиска в словаре;
	\item описать структуру разрабатываемого ПО;
	\item определить средства программной реализации;
	\item протестировать разработанное ПО;
	\item привести сведения о модулях программы;
	\item определить требования к ПО;
	\item провести экспериментальные замеры временных характеристик реализованных алгоритмов;
	\item на основании проделанной работы сделать выводы и подготовить отчет.
\end{itemize}