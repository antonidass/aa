\chapter{Технологическая часть}

В данном разделе приведены средства реализации и листинг алгоритмов.

\section{Средства реализации}

Для разработки ПО был выбран язык Java, поскольку он предоставляет разработчику широкий спектр возможностей и позволяет разрабатывать кроссплатформенные приложени, а также предоставляет большой набор инструментов для работы с многопоточностью. В качестве среды разработки использовалась Visual Studio Code. \cite{code}

\section{Листинг кода}


\begin{lstinputlisting}[
	caption={Реализация алгоритма поиска полным перебором},
	label={lst:first},
	style={javalang},
	linerange={1-15}
]{../src/Brute.java}
\end{lstinputlisting}


\begin{lstinputlisting}[
	caption={Реализация бинарного поиска},
	label={lst:second},
	style={javalang},
	linerange={1-26}
]{../src/Binary.java}
\end{lstinputlisting}


\begin{lstinputlisting}[
	caption={Реализация поиска с ипользованием частотного анализа},
	label={lst:second},
	style={javalang},
	linerange={1-60}
]{../src/Frequency.java}
\end{lstinputlisting}


\section{Тестирование функций}

В таблице~\ref{tab:tests} приведены тесты для функций, реализующих алгоритмы поиска в словаре. Тесты пройдены успешно.

\begin{table}[h!]
	\begin{center}
		\begin{tabular}{|c | c | c | c |}
            \hline
            Ключ & Словарь & Ожидание & Результат \\
            \hline
            abc & \texttt{\{car: "abc"\,, desc: "small" \}} & "small" & "small" \\
            cde & \texttt{\{car: "bva"\,, desc: "big" \}} & \text{NOT\_FOUND} & \text{NOT\_FOUND} \\
            vvv & \texttt{\{\}} & \text{NOT\_FOUND} & \text{NOT\_FOUND} \\
            \hline
		\end{tabular}
	\end{center}
	\caption{\label{tab:tests} Тестирование функций.}
\end{table}
 
\section*{Вывод}

Правильный выбор инструментов разработки позволил эффективно реализовать алгоритмы и выполнить исследовательский раздел лабораторной работы.
