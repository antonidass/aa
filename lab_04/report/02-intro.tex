\chapter*{Введение}
\addcontentsline{toc}{chapter}{Введение}

Многопоточность — способность центрального процессора (CPU) или одного ядра в многоядерном процессоре одновременно выполнять несколько процессов или потоков, соответствующим образом поддерживаемых операционной системой. Многопоточность направлена на максимизацию использования ресурсов одного ядра, используя параллелизм на уровне потоков, а также на уровне инструкций.

Сутью многопоточности является квазимногозадачность на уровне одного исполняемого процесса, то есть все потоки выполняются в адресном пространстве процесса. Кроме этого, все потоки процесса имеют не только общее адресное пространство, но и общие дескрипторы файлов. Выполняющийся процесс имеет как минимум один (главный) поток.


К достоинствам многопоточной реализации той или иной системы можно отнести следующее:
\begin{itemize}
	\item облегчение программы посредством использования общего адресного пространства;
	\item меньшие затраты на создание потока в сравнении с процессами;
	\item повышение производительности процесса за счёт распараллеливания процессорных вычислений;
	\item если поток часто теряет кэш, другие потоки могут продолжать использовать неиспользованные вычислительные ресурсы.
\end{itemize}

Недостатки:
\begin{itemize}
	\item несколько потоков могут вмешиваться друг в друга при совместном использовании аппаратных ресурсов \cite{Nemirovsky};
	\item с программной точки зрения аппаратная поддержка многопоточности более трудоемка для программного обеспечения \cite{Olukotun};
	\item проблема планирования потоков;
	\item специфика использования. Вручную настроенные программы на ассемблере, использующие расширения MMX или AltiVec и выполняющие предварительные выборки данных, не страдают от потерь кэша или неиспользуемых вычислительных ресурсов. Таким образом, такие программы не выигрывают от аппаратной многопоточности и действительно могут видеть ухудшенную производительность из-за конкуренции за общие ресурсы.
\end{itemize}
Однако несмотря на количество недостатков, перечисленных выше, многопоточная парадигма имеет большой потенциал на сегодняшний день и при должном написании кода позволяет значительно ускорить однопоточные алгоритмы.

Целью данной работы является изучение и программная реализация многопоточности на основе алгоритма поиска простых чисел.


В рамках выполнения работы необходимо решить следующие задачи:


\begin{itemize}
	\item изучить основы параллельных вычислений;
	\item исследовать и реализовать последовательный и параллельный алгоритмы поиска простых чисел;
	\item привести схемы последовательной и параллельной реализации алгоритма поиска простых чисел;
	\item описать структуру разрабатываемого ПО;
	\item определить средства программной реализации;
	\item протестировать разработанное ПО;
	\item провести сравнительный анализ последовательной и параллельной реализации на основе экспериментальных данных;
    \item подготовить отчет по лабораторной работе.
\end{itemize}
