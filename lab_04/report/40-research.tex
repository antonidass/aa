\chapter{Исследовательская часть}

В данном разделе будет произведено сравнение реализаций (последовательная, параллельная) алгоритма поиска простых чисел.

\section{Технические характеристики}

\begin{itemize}
	\item Операционная система: Windows 10. \cite{windows}
	\item Память: 16 GiB.
	\item Процессор: Intel(R) Core(TM) i7-4700HQ CPU @ 2.40GHz. \cite{intel}
\end{itemize}


\section{Временные характеристики}

Для сравнения возьмем количество потоков равное: [1, 2, 3, ... 10]. Так как проверка на простое число считается короткой задачей, возспользуемся усреднением массового эксперимента. Для этого сложим результат работы алгоритма N раз (N >= 10), после чего поделим на N. Тем самым получим достаточно точные характеристики времени. Сравнение произведено при входном параметре n = 5000000. Результат можно увидеть на рис 4.1.


\begin{figure}
	\centering
	\begin{tikzpicture}[scale=1.5]
		\begin{axis}[
			axis lines=left,
			xlabel=Количество потоков ,
			ylabel={Время, мс},
			legend pos=north west,
			ymajorgrids=true
		]
			% \addplot table[x=threads,y=time,col sep=comma]{assets/csv/time.csv};
			\addplot table[x=threads,y=time,col sep=comma]{assets/csv/time.csv};

		\end{axis}
	\end{tikzpicture}
	\captionsetup{justification=centering}
	\caption{Зависимость времени работы алгоритма от количества потоков}
	\label{plt:cmp}
\end{figure}



\clearpage


\section*{Вывод}
В результате эксперимента было получено, что при количестве потоков = 2 время работы алгоритма поиска простых чисел на 25\% быстрее последовательного выполнения алгоритма. При количестве потоков = 6 время выполнения алгоритма на 75\% быстрее последовательного выполнения алгоритма. Также стоит отметить, что при дальнейшем увеличении количества потоков время выполнения программы меняется менее обрывисто.  

Можно сделать вывод, что для получения большей производительности нужно использовать все ядра процессора.