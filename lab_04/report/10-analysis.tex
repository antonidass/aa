\chapter{Аналитическая часть}
В данном разделе будут рассмотрены алгоритмы поиска простых чисел.

\section{Описание алгоритмов}

\subsection{"Наивный"  алгоритм поиска простых чисел}

Наиболее наивный подход к поиску простых чисел заключается в следующем. Будем брать по очереди натуральные числа $n$, начиная с двойки, и проверять их на простоту. Проверка на простоту заключается в следующем: перебирая числа $k$ из диапазона от $2$ до $n - 1$, будем делить $n$ на $k$ с остатком. Если при каком-то $k$ обнаружится нулевой остаток, значит, $n$ делится на $k$ нацело, и число $n$ составное. Если же при делении обнаруживались только ненулевые остатки, значит, число простое; в этом случае выводим его на экран. Ясно, что, получив нулевой остаток (тем самым обнаружив, что $n$ составное), следует отказаться от дальнейших проб на делимость.  

Заметим, что все простые числа, за исключением двойки, нечётные. Если обработать особо случай $n = 2$, то все последующие числа $n$ можно перебирать с шагом 2. Это даст приблизительно двукратное увеличение производительности программы.


\subsection{Оптимизированный алгоритм поиска простых чисел}

Ещё одно улучшение возникает благодаря следующему утверждению: наименьший делитель составного числа $n$ не превосходит $\sqrt{n}$. Докажем это утверждение от противного. Пускай число $k$ является наименьшим делителем $n$, причём $k > \sqrt{n}$. Тогда $n = k * l$, где $l \in N$, причём $l <= \sqrt{n}$, то есть $l$ также является делителем числа $n$, кроме того, меньшим, чем $k$, а это противоречит предположению. Всё это означает, что, перебирая потенциальные делители, можно оборвать перебор, когда $k$ достигнет $\sqrt{n}$  если до этого момента делителей не найдено, то их нет вообще. Например при проверке на простоту числа 11 то наблюдение позволяет сократить перебор более чем в три раза, а для числа 1111111111111111111 — более чем в 1054092553 раза (оба числа — простые).   

Можно сделать вывод что оптимизированный алгоритм поиска простых чисел работает в несколько раз быстрее "наивного" алгоритма поиска простых чисел, следовательно в данной лабораторной работе будет рассмотрена реализация оптимизировнного алгоритма поиска простых чисел. 

\subsection{Параллельная реализация алгоритма поиска простых чисел}

Поскольку для нахождения всех простых чисел в диапазоне от 2 до $n$ нужно проверить каждое число отдельно, можно распараллелить обработку обработку $n$ чисел, разбив диапазон на несколько частей. 

\section*{Вывод} 
В данной работе стоит задача реализации последовательного и параллельного алгоритма поиска простых чисел. Входными данными будет являтся число $n$ - количество чисел, которые необходимо проверить на простоту начиная с 2. Выходными данными будет являтся массив всех простых чисел в диапазоне от 2 до $n$. В связи с ограничениями накладываемыми на ПО, входное число $n$ должно быть корректным. Необходимо дать теоретическую оценку последовательной и параллельной реализации алгоритма поиска простых чисел. 
