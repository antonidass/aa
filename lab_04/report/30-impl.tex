\chapter{Технологическая часть}

В данном разделе приведены средства реализации и листинг алгоритмов.

\section{Средства реализации}

Для разработки ПО был выбран язык Java, поскольку он предоставляет разработчику широкий спектр возможностей и позволяет разрабатывать кроссплатформенные приложени, а также предоставляет большой набор инструментов для работы с многопоточностью. В качестве среды разработки использовалась Visual Studio Code. \cite{code}

\section{Листинг кода}

\begin{lstinputlisting}[
	caption={Создание и запуск потоков},
	label={lst:simple1},
	style={javalang},
	linerange={13-37}
]{../src/project/src/main/java/Main.java}
\end{lstinputlisting}

\begin{lstinputlisting}[
	caption={Алгоритм поиска простых чисел},
	label={lst:vino1},
	style={javalang},
	linerange={1-30}
]{../src/project/src/main/java/Prime.java}
\end{lstinputlisting}


\begin{lstinputlisting}[
	caption={Метод потока поиска простых чисел},
	label={lst:vino2},
	style={javalang},
	linerange={1-30}
]{../src/project/src/main/java/PrimeNumbersThread.java}
\end{lstinputlisting}

\section{Тестирование функций}

В таблице~\ref{tabular:test_rec} приведены тесты для функций, реализующих алгоритм поиска простых чисел. Тесты пройдены успешно.

\begin{table}[h!]
	\begin{center}
		\begin{tabular}{c@{\hspace{7mm}}c@{\hspace{7mm}}c@{\hspace{7mm}}c@{\hspace{7mm}}c@{\hspace{7mm}}c@{\hspace{7mm}}}
			\hline
			Входное число n &Количество потоков &Ожидаемый результат \\ \hline
			\vspace{4mm}
			10
			&
			4
			&
			$\begin{pmatrix}
			2 & 3 & 5 & 7 
			\end{pmatrix}$ \\
			\vspace{2mm}
			\vspace{2mm}
			2
			&
			1 
			&
			2
			\\
			\vspace{2mm}
			\vspace{2mm}
			20
			&
			2
			&
			$\begin{pmatrix}
			2 & 3 & 5 & 7 & 11 & 13 & 17 & 19
			\end{pmatrix}$ \\
			\vspace{2mm}
			\vspace{2mm}
			-100 &
			4 &
			Входное число n - некорректно
		\end{tabular}
	\end{center}
	\caption{\label{tabular:test_rec} Тестирование функций}
\end{table}


\section*{Вывод}

Правильный выбор инструментов разработки позволил эффективно реализовать алгоритмы, настроить модульное тестирование и выполнить исследовательский раздел лабораторной работы.
