\chapter{Аналитическая часть}
В данном разделе будут рассмотрены алгоритмы нахождения \cite{Levenshtein} редакционного расстояния, используемые в данной лабораторной работе.  

Для преобразования одного слова в другое используются следующие операции:

\begin{itemize}
	\item D - удаление
	\item I - вставка
	\item R - замена
\end{itemize}

Будем считать стоимость каждой вышеизложенной операции - 1.  
Введем понятие совпадения - M. Его стоимость будет равна - 0.



\section{Расстояние Левенштейна}

Расстояние Левенштейна между двумя строками a и b может быть вычислено по формуле:
\begin{equation}
	\label{eq:D}
	D(i, j) = \begin{cases}
		0 &\text{i = 0, j = 0}\\
		i &\text{j = 0, i > 0}\\
		j &\text{i = 0, j > 0}\\
		\min \lbrace \\
			\qquad D(i, j-1) + 1\\
			\qquad D(i-1, j) + 1 &\text{i > 0, j > 0}\\
			\qquad D(i-1, j-1) + m(a[i], b[j]) &\text(\ref{eq:m})\\
		\rbrace
	\end{cases},
\end{equation}

где функция \ref{eq:m} определена как:
\begin{equation}
	\label{eq:m}
	m(a, b) = \begin{cases}
		0 &\text{если a = b,}\\
		1 &\text{иначе}
	\end{cases}.
\end{equation}


\section{Расстояние Дамерау — Левенштейна}
В расстоянии Дамерау - Левенштейна задействуют еще одну операцию - транспозицию T.
Расстояние Дамерау — Левенштейна может быть найдено по формуле \ref{eq:d}.
\begin{equation}
	\label{eq:d}
	D_{a,b}(i, j) = \begin{cases}
		\max(i, j), &\text{если }\min(i, j) = 0,\\
		\min \lbrace \\
			\qquad D_{a,b}(i, j-1) + 1,\\
			\qquad D_{a,b}(i-1, j) + 1,\\
			\qquad D_{a,b}(i-1, j-1) + m(a[i], b[j]), &\text{иначе}\\
			\qquad \left[ \begin{array}{cc}D_{a,b}(i-2, j-2) + 1, &\text{если }i,j > 1;\\
			\qquad &\text{}a[i] = b[j-1]; \\
			\qquad &\text{}b[j] = a[i-1]\\
			\qquad \infty, & \text{иначе}\end{array}\right.\\
		\rbrace
		\end{cases},
\end{equation}


\section*{Вывод}

Формулы Левенштейна и Дамерау — Левенштейна для рассчета расстояния между строками задаются рекурсивно, а следовательно, алгоритмы могут быть реализованы рекурсивно или итерационно. Входными данными являются две строки на русском или английском языке в любом регистре. Выходными данными является целое число — искомое расстояние для всех четырех методов и матрицы расстояний для всех методов, за исключением рекурсивного. В связи с ограничениями на ПО, входные данные должны быть корректными.
