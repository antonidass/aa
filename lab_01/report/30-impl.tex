\chapter{Технологическая часть}

В данном разделе приведены требования к программному обеспечению, выбор ЯП и листинги кода.

\section{Средства реализации}
Для разработки ПО был выбран язык Java \cite{java}, поскольку он предоставляет разработчику широкий спектр возможностей и позволяет разрабатывать кроссплатформенные приложения.  В качестве среды разработки была выбрана IntelliJ IDEA. IntelliJ IDEA \cite{vscode} подходит не только для Windows, но и для Linux.




\section{Листинг кода}

В листингe \ref{lst:algorithms} приведена реализация алгоритмов нахождения расстояния Левенштейна и Дамерау — Левенштейна, а также вспомогательные функции.

\begin{lstinputlisting}[
	caption={Листинг с алгоритмами},
	label={lst:algorithms},
    style={Java}
]{../src/lib/Levenshtein.java}
\end{lstinputlisting}
\clearpage

В таблице \ref{tabular:functional_test} приведены функциональные тесты для алгоритмов вычисления расстояния Левенштейна и Дамерау — Левенштейна. Все тесты пройдены успешно.

\begin{table}[h]
	\begin{center}
		\caption{\label{tabular:functional_test} Функциональные тесты}
		\begin{tabular}{|c|c|c|c|}
			\hline
			                    &                    & \multicolumn{2}{c|}{\bfseries Ожидаемый результат}    \\ \cline{3-4}
			\bfseries Строка 1  & \bfseries Строка 2 & \bfseries Левенштейн & \bfseries Дамерау — Левенштейн
			\csvreader{inc/csv/functional-test.csv}{}
			{\\\hline \csvcoli&\csvcolii&\csvcoliii&\csvcoliv}
			\\\hline
		\end{tabular}
	\end{center}
\end{table}


\section*{Вывод}

В данном разделе были рассмотрены листинги кода, был выбран язык программирования и среда разработки, а также было произведено функциональное тестирование. Сравнивая листинги программ видно, что написание рекуррентных подпрограмм значительно проще, чем матричных.
