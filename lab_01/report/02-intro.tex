\chapter*{Введение}
\addcontentsline{toc}{chapter}{Введение}

В данной лабораторной работе будет рассмотрен алгоритм под названием "расстояние Левенштейна". 

Данное расстояние показывает минимальное количество редакторских операций (вставки, замены и удаления), которые необходимы для перевода одной строки в другую. Это расстояние помогает определить схожесть двух строк.  

Расстояние Левенштейна используется в компьютерной лингвистике для:
\begin{itemize}
	\item исправления ошибок в слове
	\item сравнения текстовых файлов утилитой diff
	\item в биоинформатике для сравнения генов и белков
\end{itemize}

Однако кроме упомянутых трех ошибок (вставка лишнего символа, пропуск символа, замена одного символа другим), пользователь может нажимать на нужные клавиши не в том порядке. С этой проблемой поможет справится расстояние Дамерау-Левенштейна. Данное расстояние задействует еще одну редакторскую операцию - транспозицию.

Целью данной работы является анализ и реализация алгоритма Дамерау-Левенштейна и Левенштейна.  
\newline


Для достижения поставленной цели необходимо выполнить следующие задачи:
\begin{itemize}
    \item ислледовать и сравнить алгоритмы нахождения редакционного расстояния (алгоритмы Левенштейна и Дамерау--Левенштейна);
    \item привести схемы рассматриваемых алгоритмов;
    \item описать используемые структуры данных;
    \item оценить объем памяти для хранения данных;
    \item описать структуру разрабатываемого ПО;
    \item определить средства прогарммной реализации;
    \item протестировать разработанное ПО;
    \item провести сравнительный анализ алгоритмов на основе экспериментальных;
	\item подготовить отчет по лабораторной работе.
\end{itemize}
