\chapter{Исследовательская часть}

В данном разделе будет произведено сравнение последовательной реализации алгоритма кодирования строк и конвейера с использованием многопоточности, а также будет приведена демонстрация работы программы.

\section{Демонстрация работы программы}
На вход программе подается количество строк. Вывод программы представлен на рисунке 4.1 .

\img{140mm}{work_prog}{Результат работы программы}



\section{Технические характеристики}

\begin{itemize}
	\item Операционная система: Windows 10. \cite{windows}
	\item Память: 16 GiB.
	\item Процессор: Intel(R) Core(TM) i7-4700HQ CPU @ 2.40GHz. \cite{intel}
\end{itemize}


\section{Временные характеристики}

Для сравнения возьмем количество строк равное: [5, 10, 15, ... 100]. Так как кодирование исходной строки считается короткой задачей, возспользуемся усреднением массового эксперимента. Для этого сложим результат работы алгоритма N раз (N >= 10), после чего поделим на N. Тем самым получим достаточно точные характеристики времени. Результат сравнения последовательной реализации трех алгоритмов кодирования строки и конвейера с использованием многопоточности представлен на рис 4.1.




\begin{figure}
	\centering
	\begin{tikzpicture}[scale=1.5]
		\begin{axis}[
			axis lines=left,
			xlabel=Количество строк,
			ylabel={Время, с},
			legend pos=north west,
			ymajorgrids=true
		]
			\addplot table[x=n,y=lin,col sep=comma]{assets/csv/time.csv};
			\addplot table[x=n,y=conv,col sep=comma]{assets/csv/time.csv};

			\legend{Линейная реализация, Конвейерная реализация}
		\end{axis}
	\end{tikzpicture}
	\captionsetup{justification=centering}
	\caption{Зависимость времени работы алгоритмов от количества исходных строк}
	\label{plt:cmp}
\end{figure}

\clearpage


\section*{Вывод}

В результате эксперимента было получено, что время исполнения конвейерной реализации значительно меньше, чем исполнение последовательной реализации. В начале последовательная реализация выигрывает по времени, поскольку тратится время на создание потоков и ожидание доступа к переменной, что значительно снижает работоспособность. Так при количестве строк = 100 конвейерная реализация алгоритма кодирования строк быстрее последовательного выполнения алгоритма приблизительно в 2.5 раза.  
Можно сделать вывод, что для получения большей производительности стоит использовать конвейерную реализацию алгоритма.