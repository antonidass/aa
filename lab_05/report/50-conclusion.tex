\chapter*{Заключение}
\addcontentsline{toc}{chapter}{Заключение}

В рамках выполнения данной лабораторной работы были достигнуты следующие цели:
\begin{itemize}
	\item исследованы основы конвейерных вычислений;
	\item исследованы основные методы организации конвейерных вычислений;
	\item проведено сравнение существующих методов организации конвейерных вычислений;
	\item определены средства программной реализации;
	\item определены требования к ПО;
	\item приведены сведения о модулях ПО;
	\item приведены экспериментальные замеры временных характеристик реализованного конвейера;
    \item подготовлен отчет по проделанной работе.
\end{itemize}

Конвейерная обработка данных - полезный инструмент, который уменьшает время выполнения программы за счет параллелльной обработки данных. Метод дает выйгрыш по времени в том случае, когда выполняемые задачи намного больше по времени, чем время, затрачиваемое на реализацию конвейера.