\chapter{Конструкторская часть}
В данном разделе будут рассмотрены схемы рабочего и главного процессов, описаны способы тестирования, и определены структуры данных. 
\section{Разработка алгоритмов}

\img{170mm}{main}{Схема работы главного процесса}


\clearpage
\img{190mm}{proc}{Cхема рабочего процесса}


\section{Описание структур данных}
Исходя из условия задачи приходим к выводу о том, что для хранения объектов на каждой ступени конвейера необходимо использовать структуру данных - очередь. Для хранения слов, которые необходимо закодировать наиболее удобно использовать структуру данных - массив строк.
  

\section{Структура ПО}
ПО будет состоять из следующих модулей:

\begin{itemize}
    \item Основной модуль;
    \item Модуль включающий в себя реализацию алгоритма кодирования строки;
    \item Модуль для работы со ступенями конвейера;
    \item Модуль для работы с входными и выходными данными;
    \item Модуль логирования;
\end{itemize}

\section*{Вывод}

На основе теоретических данных, полученных из аналитического раздела, были построены схемы главного и рабочего процессов. Приведена структура ПО, описаны структуры данных.