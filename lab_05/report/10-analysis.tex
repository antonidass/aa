\chapter{Аналитическая часть}
В данном разделе будут рассмотрены алгоритмы: конвейерная обработка данных, кодирование строки.

\section{Описание алгоритмов}

\subsection{Описание конвейерной обработки данных}

Конвейеризация\cite{conveyor} - это техника, в результате которой задача или команда разбивается на некоторое число подзадач, которые выполняются последовательно. Каждая подкоманда выполняется на своем логическом устройстве. Все логические устройства (ступени) соединяются последовательно таким образом, что выход i-ой ступени связан с входом (i + 1)-ой ступни, все ступени работают одновременно. Множество ступеней называется конвейером. Выигрыш во времени достигается при выполнении нескольких задач за счет параллельной работы ступеней, вовлекая на каждом такте новую задачу или команду.  
В конвейере различают r последовательных этапов, так что когда i-я операция проходит s-й этап, то (i + k)-я операция проходит (s - k)-й этап.

\subsection{Алгоритм кодирования строки}
В данной лабораторной работе реализована некая функция кодирования строки, которая состоит из трех последовательных действий: применения функции шифра Цезаря\cite{ceusar}, применения функции, меняющей регистр на противоположный, применение функции, меняющей местами элементы стоящие на n / 2 символов друг от друга, где n - размер строки.
Если необходимо закодировать массив строк, то можно использовать конвейерную обработку данных. Таким образом задача будет решена эффективнее, чем при последовательном применении алгоритмов к массиву значений.  
Конвейер будет состоять из четырех уровней. Обработанные данные передаются последовательно с одного уровня (одной ленты) конвейера на следующий (следующую ленту). Далее на каждом  уровне осуществляется обработка данных. Для каждой ленты создается своя очередь задач, в которой хранятся все необработанные строки. На последнем уровне конвейера обработанные объекты попадают в пул обработанных задач.  

Уровни конвейера:
\begin{enumerate}
    \item Применение шифра Цезаря к строкам из первой очереди, запись результата во 2 очередь;
    \item Применение функции, меняющей регистры символов к строкам 2 очереди, запись результата в 3 очередь;
    \item Применение функции, меняющей местами символы в строке по определенному выше закону к строкам в 3 очереди, запись результата в пул обработанных задач.
\end{enumerate}
Поскольку запись в очередь и извлечение из очереди это не атомарные операции, необходимо создать их таковыми, путем использования мьютексов (по одному на очередь), чтобы избежать ошибок в ситуации гонок.

\section*{Вывод} 
В данной работе стоит задача реализации алгоритма конвейерной обработки данных. Входными данными будет являтся число $n$ - количество строк, которые необходимо закодировать. Выходными данными будет являтся массив закодированных строк. В связи с ограничениями накладываемыми на ПО, входное число $n$ должно быть корректным. Необходимо дать теоретическую оценку последовательной и конвейерной реализации алгоритма кодирования строк. 
