\chapter{Технологическая часть}

В данном разделе приведены средства реализации и листинг алгоритмов.

\section{Средства реализации}

Для разработки ПО был выбран язык Java, поскольку он предоставляет разработчику широкий спектр возможностей и позволяет разрабатывать кроссплатформенные приложени, а также предоставляет большой набор инструментов для работы с многопоточностью. В качестве среды разработки использовалась Visual Studio Code. \cite{code}

\section{Листинг кода}


\begin{lstinputlisting}[
	caption={Реализация первого уровня конвейера},
	label={lst:first},
	style={javalang},
	linerange={2-25}
]{../src/ConveyorStages.java}
\end{lstinputlisting}



\begin{lstinputlisting}[
	caption={Реализация второго уровня конвейера},
	label={lst:second},
	style={javalang},
	linerange={26-49}
]{../src/ConveyorStages.java}
\end{lstinputlisting}

\begin{lstinputlisting}[
	caption={Реализация третьего уровня конвейера},
	label={lst:third},
	style={javalang},
	linerange={51-75}
]{../src/ConveyorStages.java}
\end{lstinputlisting}

\begin{lstinputlisting}[
	caption={Реализация главного рабочего процесса},
	label={lst:main},
	style={javalang},
	linerange={51-92}
]{../src/Main.java}
\end{lstinputlisting}


\begin{lstinputlisting}[
	caption={Реализация алгоритма кодирования строки},
	label={lst:log},
	style={javalang},
	linerange={2-41}
]{../src/Encryption.java}
\end{lstinputlisting}





\section{Тестирование функций}

В таблице~\ref{tabular:test_rec} приведены тесты для функций, реализующих алгоритм кодирования строки. Тесты пройдены успешно.

\begin{table}[h!]
	\begin{center}
		\begin{tabular}{c@{\hspace{7mm}}c@{\hspace{7mm}}c@{\hspace{7mm}}c@{\hspace{7mm}}c@{\hspace{7mm}}c@{\hspace{7mm}}}
			\hline
			Количество строк &Исходные строки &Ожидаемый результат \\ \hline
			\vspace{4mm}
			2
			&
			$\begin{pmatrix}
				abd & bcv 
			\end{pmatrix}$ 
			&
			$\begin{pmatrix}
			  cbe & cdw
			\end{pmatrix}$ \\
			\vspace{2mm}
			\vspace{2mm}
			-100 &
			abcdeijfladkk &
			Некорректное количество строк
		\end{tabular}
	\end{center}
	\caption{\label{tabular:test_rec} Тестирование функций}
\end{table}

 
\section*{Вывод}

Правильный выбор инструментов разработки позволил эффективно реализовать алгоритмы, настроить модульное тестирование и выполнить исследовательский раздел лабораторной работы.
