\chapter{Технологическая часть}

В данном разделе приведены средства реализации и листинги кода.

\section{Требования к ПО}

К программе предъявляется ряд требований:
\begin{itemize}
	\item на вход подаётся массив сравнимых элементов;
	\item на выходе — тот же массив, но в отсортированном порядке.
\end{itemize}

\section{Средства реализации}

В данной лабораторной работе использовался язык программирования - C. \cite{kernigan} Данный язык быстр и удобен в использовании. В качестве среды разработки использовался Visual Studio Code.  \cite{vscode}

\section{Листинг кода}

В листингах \ref{lst:insertions} -- \ref{lst:bubble} приведены листинги алгоритма сортировки вставками, выбором и алгоритм быстрой сортировки соответственно. В листинге \ref{lst:benches} приведен пример реализации бенчмарков.

\begin{lstinputlisting}[
	caption={Алгоритм сортировки вставками},
	label={lst:insertions},
	style={c},
	linerange={66-81}
]{../src/lib/quick.c}
\end{lstinputlisting}

\begin{lstinputlisting}[
caption={Алгоритм сортировки выбором},
	label={lst:selection},
	style={c},
	linerange={83-97}
]{../src/lib/quick.c}
\end{lstinputlisting}



\begin{lstinputlisting}[
	caption={Алгоритм быстрой сортировки},
	label={lst:bubble},
	style={c},
	linerange={26-63}
]{../src/lib/quick.c}
\end{lstinputlisting}


\begin{lstinputlisting}[
	caption={Пример реализации бенчмарка},
	label={lst:benches},
	style={c},
	linerange={100-109}
]{../src/lib/algorithms/quick.c}
\end{lstinputlisting}



\section{Тестирование функций}

В таблице~\ref{tbl:test} приведены тесты для функций, реализующих алгоритмы сортировки. Тесты пройдены успешно.

\begin{table}[h!]
	\begin{center}
		\begin{tabular}{|c|c|c|}
			\hline
			Входной массив & Ожидаемый результат & Результат \\ 
			\hline
			$[1,2,3,4]$ & $[1,2,3,4]$  & $[1,2,3,4]$\\
			$[3,2,1]$  & $[1,2,3]$ & $[1,2,3]$\\
			$[5,6,2,4,-2]$  & $[-2,2,4,5,6]$  & $[-2,2,4,5,6]$\\
			$[4]$  & $[4]$  & $[4]$\\
			$[]$  & $[]$  & $[]$\\
			\hline
		\end{tabular}
		\caption{\label{tbl:test}Тестирование функций}
	\end{center}
\end{table}

\section*{Вывод}

Правильный выбор инструментов разработки позволил эффективно реализовать алгоритмы, настроить модульное тестирование и выполнить исследовательский раздел лабораторной работы.
