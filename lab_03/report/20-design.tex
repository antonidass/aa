\chapter{Конструкторская часть}

В данном разделе будут рассмотрены схемы алгоритмов сортировки.

\section{Разработка алгоритмов}


\img{170mm}{quick}{Схема алгоритма быстрой сортировки}
\img{170mm}{choice1}{Схема алгоритма сортировки выбором}
\clearpage
\img{170mm}{insert}{Схема алгоритма сортировки вставками}

\section{Модель вычислений}

Для последующего вычисления трудоемкости необходимо ввести модель вычислений:
\begin{enumerate}
    \item операции из списка (\ref{for:opers}) имеют трудоемкость 1;
        \begin{equation}
            \label{for:opers}
            +, -, /, *, \%, =, +=, -=, *=, /=, \%=, ==, !=, <, >, <=, >=, [], ++, {-}-
        \end{equation}
    \item трудоемкость оператора выбора \code{if условие then A else B} рассчитывается, как (\ref{for:if});
	\begin{equation}
        \label{for:if}
        f_{if} = f_{\text{условия}} +
        \begin{cases}
        f_A, & \text{если условие выполняется,}\\
        f_B, & \text{иначе.}
        \end{cases}
	\end{equation}
\item трудоемкость цикла рассчитывается, как (\ref{for:for});
    \begin{equation}
        \label{for:for}
        f_{for} = f_{\text{инициализации}} + f_{\text{сравнения}} + N(f_{\text{тела}} + f_{\text{инкремента}} + f_{\text{сравнения}})
    \end{equation}
	\item трудоемкость вызова функции равна 0.
\end{enumerate}


\section{Трудоёмкость алгоритмов}

Обозначим во всех последующих вычислениях размер массивов как N.

\subsection{Алгоритм сортировки выбором}

Трудоёмкость алгоритма сортировки выбором состоит из:
\begin{itemize}
    \item Трудоёмкость сравнения, инкремента внешнего цикла, а также зависимых только от него операций, по $i \in [1..N)$, которая равна (\ref{for:selection_outer}):
        \begin{equation}
            \label{for:selection_outer}
            f_{outer} = 2 + 12(N - 1)
        \end{equation}
    \item Суммарная трудоёмкость внутренних циклов, количество итераций которых меняется в промежутке $[1..N-1]$, которая равна (\ref{for:selection_inner}):
        \begin{equation}
            \label{for:selection_inner}
            f_{inner} = 2(N - 1) + \frac{N \cdot (N - 1)}{2} \cdot f_{if}
        \end{equation}
    \item Трудоёмкость условия во внутреннем цикле, которая равна (\ref{for:selection_if}):
        \begin{equation}
            \label{for:selection_if}
            f_{if} = 3 + \begin{cases}
                0, & \text{л.с.}\\
                3, & \text{х.с.}\\
            \end{cases}
        \end{equation}
\end{itemize}

Трудоёмкость в лучшем случае (\ref{for:selection_best}):
\begin{equation}
    \label{for:selection_best}
    f_{best} = -12 + 12.5N + \frac{3}{2}N^2 \approx \frac{3}{2}N = O(N^2)
\end{equation}

Трудоёмкость в худшем случае (\ref{for:selection_worst}):
\begin{equation}
    \label{for:selection_worst}
    f_{worst} = -12 + 11N + 3N^2 \approx 3N^2 = O(N^2)
\end{equation}



\subsection{Алгоритм быстрой сортировки}

Трудоёмкость в лучшем случае (\ref{for:bubble_best}):
\begin{equation}
    \label{for:bubble_best}
    f_{best} = O(n * log(N))
\end{equation}

Трудоёмкость в худшем случае (\ref{for:bubble_worst}):
\begin{equation}
    \label{for:bubble_worst}
    f_{worst} = O(N^2)
\end{equation}



\subsection{Алгоритм сортировки вставками}

Трудоёмкость в лучшем случае (\ref{for:insert_best}):
\begin{equation}
    \label{for:insert_best}
    f_{best} = O(n)
\end{equation}

Трудоёмкость в худшем случае (\ref{for:insert_worst}):
\begin{equation}
    \label{for:insert_worst}
    f_{worst} = O(N^2)
\end{equation}

\subsection{Память используемая алгоритмом}

Сортировки выбором и вставками не используют дополнительных структур, поэтому затраченная память будет равна O(n)
Быстрая сортировка использует O(log(n)) дополнительной памяти, хранящейся в стеке. Полная затраченная память будет равна O(n) + O(log(n)).


\subsection{Описание структур данных}

Для реализации алгоритмов сортировки использовалась структура данных - массив. Также можно было сортировать связный список, однако при сортировке данной структуры потребуется гораздо больше времени, поэтому был выбран массив.

\subsection{Описание способов тестирования}

Данные сортировки можно протестировать функционально. При этом можно выделить следующие классы эквивалентности: 

\begin{itemize}
    \item Упорядоченный массива;
    \item Массив одинаковых элементов;
    \item Неупорядоченный массив;
    \item Массив из одного элемента;
\end{itemize}


\section*{Вывод}

Были разработаны схемы всех трех алгоритмов сортировки.
Для каждого из них были рассчитаны и оценены лучшие и худшие случаи.
