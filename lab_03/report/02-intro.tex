\chapter*{Введение}
\addcontentsline{toc}{chapter}{Введение}

В данной лабораторной работе будут рассмотрены алгоримты сортировки.

Алгоритмы сортировки используются практически в любой программной системе. Целью алгоритмов сортировки является упорядочение последовательности элементов данных. Поиск элемента в последовательности отсортированных данных занимает время, пропорциональное логарифму количеству элементов в последовательности, а поиск элемента в последовательности не отсортированных данных занимает время, пропорциональное количеству элементов в последовательности, то есть намного больше. Существует множество различных методов сортировки данных. Однако любой алгоритм сортировки можно разбить на три основные части:
\begin{itemize}
    \item сравнение, определяющее упорядоченность пары элементов;
    \item перестановка, меняющая местами пару элементов;
    \item собственно сортирующий алгоритм, который осуществляет сравнение и перестановку элементов данных до тех пор, пока все эти элементы не будут упорядочены.
\end{itemize}

Целью данной работы является изучение трех алгоритмов сортировки и реализации данных алгоритмов.

В рамках выполнения работы необходимо решить следующие задачи:
\begin{itemize}
	\item исследовать и сравнить 3 алгоритма сортировки: вставками, выбором и алгоритм быстрой сортировки;
	\item привести схемы рассматриваемых алгоритмов (вставками, выбором, алгоритм быстрой сортировки);
	\item описать используемые структуры данных;
	\item оценить объем памяти для хранения данных;
	\item описать структуру разрабатываемого ПО.
	\item определить средста программной реализации;
	\item протестировать разработанное ПО.
	\item провести сравнительный анализ алгоритмов на основе экспериментальных данных;
    \item подготовить отчет по лабораторной работе.
\end{itemize}
