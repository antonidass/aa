\chapter{Аналитическая часть}

\section{Быстрая сортировка}

В алгоритме быстрой сортировки (Quicksort) используется рекурсивный подход.
Выбрав опорный элемент в списке данный алгоритм сортировки делит список на две части, относительно выбранного элемента.
Далее в первую часть попадают все элементы, меньшие выбранного, а во вторую — большие элементы. 
Если в данных частях более двух элементов, рекурсивно запускается для него та же процедура. 
В конце получится полностью отсортированная последовательность.

\section{Сортировка вставками}

Сортировка вставками — алгоритм сортировки, котором элементы входной последовательности просматриваются по одному, и каждый новый поступивший элемент размещается в подходящее место среди ранее упорядоченных элементов \cite{Knut}.

В начальный момент отсортированная последовательность пуста.
На каждом шаге алгоритма выбирается один из элементов входных данных и помещается на нужную позицию в уже отсортированной последовательности до тех пор, пока набор входных данных не будет исчерпан.
В любой момент времени в отсортированной последовательности элементы удовлетворяют требованиям к выходным данным алгоритма.

\section{Сортировка выбором}

Шаги алгоритма:
\begin{enumerate}
	\item находим номер минимального значения в текущем массиве;
	\item производим обмен этого значения со значением первой неотсортированной позиции (обмен не нужен, если минимальный элемент уже находится на данной позиции);
	\item теперь сортируем ``хвост'' массива, исключив из рассмотрения уже отсортированные элементы.
\end{enumerate}

Для реализации устойчивости алгоритма необходимо в пункте 2 минимальный элемент непосредственно вставлять в первую неотсортированную позицию, не меняя порядок остальных элементов, что может привести к резкому увеличению числа обменов. 

\section*{Вывод}
В данной работе стоит задача реализации 3 алгоритмов сортировки, а именно: вставками, выбором и быстрой сортировки.
Необходимо оценить теоретическую оценку алгоритмов и проверить ее экспериментально.
