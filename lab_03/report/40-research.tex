\chapter{Исследовательская часть}

В данном разделе будет произведено сравнение вышеизложенных алгоритмов.

\section{Технические характеристики}

\begin{itemize}
	\item Операционная система: Windows 10 64 bit. \cite{windows}
	\item Память: 16 GiB.
	\item Процессор: Intel(R) Core(TM) i7-4700HQ CPU @ 2.40GHz \cite{intel}.
\end{itemize}

Тестирование проводилось на ноутбуке, включенном в сеть электропитания. Во время тестирования ноутбук был нагружен только встроенными приложениями окружения, окружением, а также непосредственно системой тестирования.

\section{Время выполнения алгоритмов}

Для сравнения возьмем массивы размерностью [100, 200, 300, ... 1000]. Так как подсчет сортировки массива считается короткой задачей, воспользуемся усреднением массового эксперимента. Для этого сложим результат работы алгоритма n раз, после чего поделим на n. Тем самым достаточно точные характеристики времени. Сравнение произведем при n = 100. 


\img{130mm}{1}{Время работы алгоритмов на лучших данных}
\clearpage
\img{130mm}{3}{Время работы алгоритмов на худших данных}
\clearpage
\img{130mm}{2}{Время работы алгоритмов на рандомных данных}


\section*{Вывод}
В результате эксперимента было получено, что на случайных данных, алгоритм быстрой сортировки работает быстрее чем алгоритм выбором и вставками. Например при размерности массива равной 900 алгоритм быстрой сортировки работает в 8 раз быстрее алгоритма сортировки выбором и в 5.5 раз быстрее алгоритма вставками. 