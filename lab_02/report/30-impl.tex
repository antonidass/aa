\chapter{Технологическая часть}

В данном разделе приведены средства реализации и листинг кода.

\section{Требования к ПО}

К программе предъявляется ряд требований:
\begin{itemize}
	\item на вход подаются размеры 2 матриц, а также их элементы;
	\item на выходе — матрица, которая является результатом умножения входных матриц.
\end{itemize}

\section{Средства реализации}

В качестве языка программирования был выбран ЯП Java. Данный выбор обусловлен тем, что данный язык легок в использовании, быстр и поддерживает кроссплатформенность.

\section{Листинг кода}

\begin{lstinputlisting}[
	caption={Стандартный алгоритм умножения матриц},
	label={lst:simple},
	style={javalang},
	linerange={5-24}
]{../src/lib/Main.java}
\end{lstinputlisting}

\begin{lstinputlisting}[
	caption={Алгоритм Копперсмита — Винограда},
	label={lst:vino},
	style={javalang},
	linerange={26-69}
]{../src/lib/Main.java}
\end{lstinputlisting}

\section{Тестирование функций}

В таблице~\ref{tabular:test_rec} приведены тесты для функций, реализующих стандартный алгоритм умножения матриц и алгоритм Винограда. Тесты пройдены успешно.

\begin{table}[h!]
	\begin{center}
		\begin{tabular}{c@{\hspace{7mm}}c@{\hspace{7mm}}c@{\hspace{7mm}}c@{\hspace{7mm}}c@{\hspace{7mm}}c@{\hspace{7mm}}}
			\hline
			Матрица 1 & Матрица 2 &Ожидаемый результат \\ \hline
			\vspace{4mm}
			$\begin{pmatrix}
			1 & 1 & 1\\
			2 & 2 & 2\\
			3 & 3 & 3
			\end{pmatrix}$ &
			$\begin{pmatrix}
			2 & 2 & 2\\
			3 & 3 & 3\\
			1 & 1 & 1
			\end{pmatrix}$ &
			$\begin{pmatrix}
			6 & 6 & 6\\
			12 & 12 & 12\\
			18 & 18 & 18
			\end{pmatrix}$ \\
			\vspace{2mm}
			\vspace{2mm}
			$\begin{pmatrix}
            1 & 5 & 5\\
            1 & 5 & 5
			\end{pmatrix}$ &
			$\begin{pmatrix}
			1 & 3\\
			1 & 3\\
            1 & 3
			\end{pmatrix}$ &
			$\begin{pmatrix}
			11 & 33\\
			11 & 33
			\end{pmatrix}$ \\
			\vspace{2mm}
			\vspace{2mm}
			$\begin{pmatrix}
			2
			\end{pmatrix}$ &
			$\begin{pmatrix}
			2
			\end{pmatrix}$ &
			$\begin{pmatrix}
			4
			\end{pmatrix}$ \\
			\vspace{2mm}
			\vspace{2mm}
			$\begin{pmatrix}
			1 & -2 & 3\\
			1 & 2 & 3\\
			1 & 2 & 3
			\end{pmatrix}$ &
			$\begin{pmatrix}
			-1 & 2 & 3\\
			1 & 2 & 3\\
			1 & 2 & 3
			\end{pmatrix}$ &
			$\begin{pmatrix}
			0 & 4 & 6\\
			4 & 12 & 18\\
			4 & 12 & 18
			\end{pmatrix}$\\
			\vspace{2mm}
			\vspace{2mm}
			$\begin{pmatrix}
			111 & 222
			\end{pmatrix}$ &
			$\begin{pmatrix}
			4 & 0
			\end{pmatrix}$ &
			Не могут быть перемножены\\
		\end{tabular}
	\end{center}
	\caption{\label{tabular:test_rec} Тестирование функций}
\end{table}


\section{Оптимизация алгоритма Копперсмита - Винограда}
Пусть в дальнейшем K - общий размер при умножении матриц размеров $M \times K$ и $K \times N$.

Видно, что для алгоритма Винограда худшим случаем являются матрицы с нечётным общим размером, а лучшим - с чётным, т. к. отпадает необходимость в последнем цикле.

Данный алгоритм можно оптимизировать:
\begin{itemize}
	\item заменой операции деления на 2 побитовым сдвигом на 1 вправо;
	\item заменой выражения вида \code{a = a + ...} на \code{a += ...};
	\item сделав в циклах по k шаг 2, избавившись тем самым от двух операций умножения на каждую итерацию.
\end{itemize}


\section*{Вывод}

Правильный выбор инструментов разработки позволил эффективно реализовать алгоритмы, настроить модульное тестирование и выполнить исследовательский раздел лабораторной работы.
