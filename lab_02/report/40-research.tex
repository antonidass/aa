\chapter{Исследовательская часть}

В данном разделе будет произведено сравнение вышеизложенных алгоритмов.

\section{Технические характеристики}

\begin{itemize}
	\item Операционная система: Windows 10.
	\item Память: 16 GiB.
	\item Процессор: Intel(R) Core(TM) i7-4700HQ CPU @ 2.40GHz.
\end{itemize}


\section{Временные характеристики}

Для сравнения возьмем квадратные матрицы размерностью [10, 20, 30, ... 100]. Так как подсчет умножения матриц считается короткой считается задачей, возспользуемся усреднением массового эксперимента. Для этого сложим результат работы алгоритма n раз (n >= 10), после чего поделим на n. Тем самым получим достаточно точные характеристики времени. Сравнение произведен при n = 50. Результат можно увидеть на рис 4.1.

\clearpage
\img{120mm}{time1}{Временные характеристики на четных размерах матриц}
\clearpage
\img{120mm}{time2}{Временные характеристики на нечетных размерах матриц}


\section*{Вывод}

В данном разделе было произведено сравнение количества затраченного времени вышеизложенных алгоритмов. Самым быстрым оказался модифицированный алгоритм Винограда. При этом в алгоритме Винограда умножения матриц требуется дополнительно m + n памяти под результат.