\chapter{Исследовательская часть}

В данном разделе будет произведено сравнение вышеизложенных алгоритмов.

\section{Технические характеристики}

\begin{itemize}
	\item Операционная система: Windows 10. \cite{windows}
	\item Память: 16 GiB.
	\item Процессор: Intel(R) Core(TM) i7-4700HQ CPU @ 2.40GHz. \cite{intel}
\end{itemize}


\section{Временные характеристики}

Для сравнения возьмем квадратные матрицы размерностью [10, 20, 30, ... 100]. Так как подсчет умножения матриц считается короткой считается задачей, возспользуемся усреднением массового эксперимента. Для этого сложим результат работы алгоритма n раз (n >= 10), после чего поделим на n. Тем самым получим достаточно точные характеристики времени. Сравнение произведен при n = 50. Результат можно увидеть на рис 4.1.

\clearpage
\img{120mm}{time1}{Временные характеристики на четных размерах матриц}
\clearpage
\img{120mm}{time2}{Временные характеристики на нечетных размерах матриц}


\section*{Вывод}
В результате эксперимента было получено, что на четных размерах матрицы и N ~= 100 алгоритм Винограда работает на 40\% быстрее стандартного алгоритма, при этом Алгоритм Винограда с оптимизациями работает в 5 раз быстрее стандартного алгоритма и в 2.5 раза быстрее обычного Алгоритма Винограда.   
На нечетных размерха матрицы аналогичная ситуация: при N = 100 алгоритм Винограда с оптимизациями работает в 3 раза быстрее стандартного алгоритма, при том что стандартный алгоритм работает одинаково с четными и нечетными матрицами. Можно сделать вывод, что алгоритм Винограда эффективнее работает с четными матрицами.