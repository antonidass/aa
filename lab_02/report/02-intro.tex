\chapter*{Введение}
\addcontentsline{toc}{chapter}{Введение}

Умножение матриц активно используется в компьютерной графике. В частности для того, чтобы изменить положение объекта с координатами x, y, z на некоторое смещение dx, dy, dz. В этом случае нужно умножить координаты объекта на матрицу перемещения. Аналогичная ситуация, если нужно повернуть объект. В этом случае матрциа перемещения заменяется на матрицу вращения и производится та же операция умножения матриц.   

При достаточно большом количестве объектов возникает необходимость минимизировать временные затраты на произведение матриц.

Алгоритм Копперсмита — Винограда — алгоритм умножения квадратных матриц, предложенный в 1987 году Д. Копперсмитом и Ш. Виноградом \cite{Coppersmith}.
В исходной версии асимптотическая сложность алгоритма составляла $O(n^{2,3755})$, где  $n$ — размер стороны матрицы.
Алгоритм Копперсмита — Винограда, с учетом серии улучшений и доработок в последующие годы, обладает лучшей асимптотикой среди известных алгоритмов умножения матриц \cite{Cohn}.

Целью данной работы является изучение, программная реализация, а также сравнение алгоритмов умножения матриц.


В рамках выполнения работы необходимо решить следующие задачи:

\begin{enumerate}
	\item изучить и реализовать 2 алгоритма перемножения матриц: стандартный и Копперсмита-Винограда;
	\item сравнить трудоёмкость алгоритмов на основе теоретических расчетов;
	\item сравнить алгоритмы на основе экспериментальных данных;
    \item подготовить отчет по лабораторной работе.
\end{enumerate}

